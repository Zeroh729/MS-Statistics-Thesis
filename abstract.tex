%%%%%%%%%%%%%%%%%%%%%%%%%%%%%%%%%%%%%%%%%%%%%%%%%%%%%%%%%%%%%%%%%%%%%%%%%%%%%%%%%%%%%%%%%%%%%%%%%%%%%%
%
%   Filename    : abstract.tex 
%
%   Description : This file will contain your abstract.
%                 
%%%%%%%%%%%%%%%%%%%%%%%%%%%%%%%%%%%%%%%%%%%%%%%%%%%%%%%%%%%%%%%%%%%%%%%%%%%%%%%%%%%%%%%%%%%%%%%%%%%%%%

\begin{abstract}
The digital PCR (dPCR) is an emerging technology to quantify the DNA copies of known strains related to diseases. Currently, the dPCR methodology is being further researched and improved to surpass its accuracy over the gold standard real-time qPCR. One area of study in dPCR is its "digitization" step in which droplets are classified as positives or negatives. This thesis reviews the current droplet classification methods of single-channel dPCR quantification and proposes the Expectation-Maximization Clustering method in aims to improve the accuracy of the final estimated DNA concentration.

%
%  Do not put citations or quotes in the abract.
%

\begin{flushleft}
\begin{tabular}{lp{4.25in}}
\hspace{-0.5em}\textbf{Keywords:}\hspace{0.25em} & Quantitative PCR, Droplet Digital PCR, Expectation Maximization Clustering \\
\end{tabular}
\end{flushleft}
\end{abstract}
