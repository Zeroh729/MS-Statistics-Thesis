%%%%%%%%%%%%%%%%%%%%%%%%%%%%%%%%%%%%%%%%%%%%%%%%%%%%%%%%%%%%%%%%%%%%%%%%%%%%%%%%%%%%%%%%%%%%%%%%%%%%%%
%
%   Filename    : abstract.tex 
%
%   Description : This file will contain your abstract.
%                 
%%%%%%%%%%%%%%%%%%%%%%%%%%%%%%%%%%%%%%%%%%%%%%%%%%%%%%%%%%%%%%%%%%%%%%%%%%%%%%%%%%%%%%%%%%%%%%%%%%%%%%

\begin{abstract}
The digital PCR (dPCR) is a method to quantify the DNA copies of known strains related to diseases. As a new approach to the gold-standard RT-PCR, further research is required to assess the quality and accuracy of this method. One particular area of dPCR is its novel step in droplet classification that distinguishes it from RT-PCR. As of writing, few droplet classifiers exist in literature as well as the assessment of these methods. This thesis reviews the classification methods of current dPCR quantification tools in literature, and proposes the Expectation Maximization Clustering method in aims to improve the accuracy of the final estimated DNA concentration.

%
%  Do not put citations or quotes in the abract.
%

\begin{flushleft}
\begin{tabular}{lp{4.25in}}
\hspace{-0.5em}\textbf{Keywords:}\hspace{0.25em} & Quantitative PCR, Droplet Digital PCR, Expectation Maximization Clustering \\
\end{tabular}
\end{flushleft}
\end{abstract}
